%% start of file `main.tex'.
%% Copyright 2014 Francois Mouton (moutonf@gmail.com).
%
% This template is adapted from the work performed by Xavier Danaux (xdanaux@gmail.com).
% This template further extends the functionality by integrating the moderntimeline package.
% This template also includes custom Biblatex style to print bibliography items with the moderntimeline package.
%
% This work may be distributed and/or modified under the
% conditions of the LaTeX Project Public License version 1.3c,
% available at http://www.latex-project.org/lppl/.


\documentclass[10pt,a4paper,sans]{moderncv}        % possible options include font size ('10pt', '11pt' and '12pt'), paper size ('a4paper', 'letterpaper', 'a5paper', 'legalpaper', 'executivepaper' and 'landscape') and font family ('sans' and 'roman')

% moderncv themes
\moderncvstyle{classic}                             % Only the 'classic' style is fully functional with the modifications made. The other options, 'casual' (default), 'oldstyle' and 'banking' has minor typesetting problems with the current modifications.
\moderncvcolor{black}                               % color options 'blue' (default), 'orange', 'green', 'red', 'purple', 'grey' and 'black'
%\renewcommand{\familydefault}{\sfdefault}         % to set the default font; use '\sfdefault' for the default sans serif font, '\rmdefault' for the default roman one, or any tex font name

% character encoding
\usepackage[utf8]{inputenc}                       % if you are not using xelatex ou lualatex, replace by the encoding you are using

% adjust the page margins
\usepackage[scale=0.83]{geometry}
\setlength{\hintscolumnwidth}{3cm}                % if you want to change the width of the column of the timeline
%\setlength{\makecvtitlenamewidth}{10cm}           % for the 'classic' style, if you want to force the width allocated to your name and avoid line breaks. Be careful though, the length is normally calculated to avoid any overlap with your personal info; use this at your own typographical risks.
\usepackage{ragged2e} %Lo he incluido yo para poder justificar la cover letter. Parece que no da problemas.

%-------------------Inlcuding pdfpages package-------------------------------------------------------------

\usepackage{pdfpages/pdfpages}

%-------------------Including moderntimeline package-------------------------------------------------------

\usepackage{moderntimeline/moderntimeline}

\tlmaxdates{2004}{2014}                             % Set the scale of the timeline. \tlmaxdates{startDate}{endDate}

%-------------------Including xpatch package---------------------------------------------------------------

\usepackage{xpatch/xpatch}

%-------------------Including Biblatex package-------------------------------------------------------------

\usepackage[url=false,
    backend=biber,                                  % This can be set to either biber or bibtex. If references are missing just change back and forth between biber and bibtex..
    style=authoryear,
    doi=false,  
    isbn=false,
    backref=false,
    dashed=false,                                   % Do not add a dash out authors for subsequent articles with the same authors.
    maxnames=99,                                    % Amount of authors to include before abbreviating.
    sorting=ydnt]{biblatex}                         % Sorting in reverse order

\addbibresource{cvreferences.bib}                   % Include your bibtex file here. Format: fileName.bib

\input{biblatex_modifications/standard_modification.tex}        % Modifying the default standard.tex style of Biblatex. This modification is performed to include the moderntimeline package.

%-------------------Defining a CV Reference column style and a CV reference entry block-------------------

% Adapted from the solution provided in: http://tex.stackexchange.com/questions/34881/references-section-in-a-cv
% usage: \cvreference{name}{address line 1}{address line 2}{address line 3}{address line 4}{e-mail address}{phone number}{mobile phone number}
% Everything but the name is optional
% If \addresssymbol, \emailsymbol or \phonesymbol are specified, they will be used.
% (Per default, \addresssymbol isn't specified, the other two are specified.)
% If you don't like the symbols, remove them from the following code, including the tilde ~ (e.g. \phonesymbol~).

\newcommand{\cvreferencecolumn}[2]{%
  \cvitem[0.75em]{}{%
    \begin{minipage}[t]{\listdoubleitemmaincolumnwidth}#1\end{minipage}%
    \hfill%
    \begin{minipage}[t]{\listdoubleitemmaincolumnwidth}#2\end{minipage}%
    }%
}

\newcommand{\cvreference}[8]{%
    \textbf{#1}\newline% Name
    \ifthenelse{\equal{#2}{}}{}{\addresssymbol~#2\newline}%
    \ifthenelse{\equal{#3}{}}{}{#3\newline}%
    \ifthenelse{\equal{#4}{}}{}{#4\newline}%
    \ifthenelse{\equal{#5}{}}{}{#5\newline}%
    \ifthenelse{\equal{#6}{}}{}{\emailsymbol~\texttt{\href{mailto:#6}{\nolinkurl{#6}}}\newline}%
    \ifthenelse{\equal{#7}{}}{}{\phonesymbol~#7\newline}
    \ifthenelse{\equal{#8}{}}{}{\mobilephonesymbol~#8}}

%-------------------Personal Data for CV title-----------------------------------------------------------
% Example:
%\name{\huge \textbf{Ricardo}}{\huge \textbf{Grande Izquierdo}} %Cambiar segun si quiero el CV o la carta
\name{Ricardo}{Grande}
\title{Curriculum Vitae}                               % optional, remove / comment the line if not wanted
\address{Dept. of Mathematics, Office 2842}{U. of Michigan, Ann Arbor, MI 48109}% optional, remove / comment the line if not wanted; the "postcode city" and and "country" arguments can be omitted or provided empty
\phone[mobile]{(+1)~210~362~0805}                   % optional, remove / comment the line if not wanted
%\phone[fixed]{+2~(345)~678~901}                    % optional, remove / comment the line if not wanted
%\phone[fax]{+3~(456)~789~012}                      % optional, remove / comment the line if not wanted
\email{grander@umich.edu}                               % optional, remove / comment the line if not wanted
%\homepage{math.mit.edu/\textasciitilde{}rgi/}                         % optional, remove / comment the line if not wanted
%\extrainfo{additional information}                 % optional, remove / comment the line if not wanted
%\photo[64pt][0.4pt]{images/picture}                       % optional, remove / comment the line if not wanted; '64pt' is the height the picture must be resized to, 0.4pt is the thickness of the frame around it (put it to 0pt for no frame) and 'picture' is the name of the picture file stored
%\quote{Some quote}                                 % optional, remove / comment the line if not wanted

%-------------------------------------------------------------------------------------------------------
%   Content
%-------------------------------------------------------------------------------------------------------
\begin{document}

%-------------------Resume------------------------------------------------------------------------------

\makecvtitle

%-------------------Research Interests--------------------------------------------------------------

\section{Research Interests}

% Format:  \cvitem{Section Name}{Description}
% Example: \cvitem{title}{\emph{The title of my Masters goes here}}
% Example: \cvitem{supervisors}{My supervisors' names go here}
% Example: \cvitem{description}{Short thesis abstract}

\cvitem{}{Nonlinear Dispersive PDEs, Harmonic Analysis}


%-------------------Academic Appointments-------------------------------------------------------------------

\section{Academic Appointments}

% For a date range: (To indicate 'up to present', set EndYear to 0)
% Format:  \tlcventry{StartYear}{EndYear}{Degree}{Institution}{City}{\textit{Grade}}{Description}  % Arguments 3 (Degree) to 6 (Grade) can be left empty. 
% Example: \tlcventry{2012}{0}{BSc Computer Science}{University of MyCity}{MyCity}{}{Also completed several random courses}

%\tlcventry{2012}{0}{BSc Computer Science}{University of MyCity}{MyCity}{}{Also completed several random courses}

% For a single year:
% Format:  \tldatecventry{StartYear}{Degree}{Institution}{City}{\textit{Grade}}{Description}
% Example: \tldatecventry{2008}{Senior Certificate}{High School MyCity}{MyCity}{\textit{80\%}}{Passed with distinction}

%\tldatecventry{2008}{Senior Certificate}{High School MyCity}{MyCity}{\textit{80\%}}{Passed with distinction}
\cvitem{2020 - Currently}{\textbf{Postdoctoral Assistant Professor}, \emph{University of Michigan, Ann Arbor}}
\cvlistitem{ \emph{Mentor:} Zaher Hani}


%-------------------Education Section-------------------------------------------------------------------

\section{Education}

% For a date range: (To indicate 'up to present', set EndYear to 0)
% Format:  \tlcventry{StartYear}{EndYear}{Degree}{Institution}{City}{\textit{Grade}}{Description}  % Arguments 3 (Degree) to 6 (Grade) can be left empty. 
% Example: \tlcventry{2012}{0}{BSc Computer Science}{University of MyCity}{MyCity}{}{Also completed several random courses}

%\tlcventry{2012}{0}{BSc Computer Science}{University of MyCity}{MyCity}{}{Also completed several random courses}

% For a single year:
% Format:  \tldatecventry{StartYear}{Degree}{Institution}{City}{\textit{Grade}}{Description}
% Example: \tldatecventry{2008}{Senior Certificate}{High School MyCity}{MyCity}{\textit{80\%}}{Passed with distinction}

%\tldatecventry{2008}{Senior Certificate}{High School MyCity}{MyCity}{\textit{80\%}}{Passed with distinction}
\cvitem{2015 - 2020}{\textbf{PhD  in Mathematics}, \emph{Massachusetts Institute of Technology}}
\cvlistitem{ \emph{Advisor:} Gigliola Staffilani}
\cvlistitem{ \emph{Thesis title:} The role of smoothing effect in some dispersive equations} %{ \ \emph{GPA}: 5.0/5.0.}
\cvitem{2014 - 2015}{\textbf{Master of Advanced Study in Mathematics}, \emph{University of Cambridge}}
\cvlistitem{ \emph{Essay:} Averaging Lemmas and the X-ray transform}
\cvlistitem{ \emph{Directed by:} Cl\'ement Mouhot}
\cvitem{2010-2014}{\textbf{Licenciatura en Matem\'aticas}, \emph{Universidad del Pa\'is Vasco (UPV-EHU)}}
%\cvlistitem{Valedictorian Award}


%-------------------Publications----------------------------------------------------------------

\section{Publications}

% Format:  \cvlistitem{Achievement}
% Example: \cvlistitem{Received best student award}
% Example: \cvlistitem{Another achievement. This achievement is particularly long and therefore normally spans over several lines. Did you notice the indentation when the line wraps?}

\cvitem{}{R. Grande, K. M. Kurianski, G. Staffilani, \emph{On the nonlinear Dysthe equation}, Nonlinear Analysis 207, 112292 (2021)}
\cvitem{}{R. Grande, \emph{Continuum limit for discrete NLS with memory effect}, submitted, preprint available at \url{arxiv.org/abs/1910.05681}}
\cvitem{}{R. Grande, \emph{Space-time fractional Nonlinear Schr\"odinger equation}, SIAM J.$\!$ Math.$\!$ Anal (2019), 51(5), 4172-4212}
\cvitem{}{R. Grande, I. Kov\'acs, K. Kutnar, A. Malni\v{c}, L. Mart\'inez, D. Maru\v{s}i\v{c},  \emph{Equisizable partial sum families}, Journal of Algebraic Combinatorics 51, 273-296 (2020)}
\cvitem{}{M. Conder, R. Grande, \emph{On embeddings of circulant graphs}, Electronic Journal of Combinatorics 22 (2015), $\#$ P2.28}

%-------------------Conferences and Workshops------------------------------------------------------------------

\section{Conferences and Workshops}

\cvitem{Nov 2020}{\textbf{Differential Equations Seminar}, University of Michigan (Invited speaker)}
\cvitem{May 2020}{\textbf{Mathematics of Planet Earth: Analysis and Modelling}, \href{https://sites.google.com/view/mpe2020-webinars/home}{Webinar} (Invited speaker)}
\cvitem{May 2020}{\textbf{Mathematical Questions in Wave Turbulence}, Banff International Research Station}
\cvitem{Jan 2020}{\textbf{Winter School: Turbulence in fluids and PDEs}, Lausanne (Invited speaker)}
\cvitem{Jan 2020}{\textbf{Seminar}, GSSI L'Aquila. (Invited speaker)}
\cvitem{Jan 2020}{\textbf{BCAM Scientific Seminar}, BCAM. (Invited speaker)}
\cvitem{Dec 2019}{\textbf{Meeting: Simons Collaboration in Wave Turbulence}, Courant Institute}
\cvitem{Nov 2019}{\textbf{Brown-BU-UMass Amherst seminar in PDE and Dynamics}, Brown University (Invited speaker)}
\cvitem{Nov 2018}{\textbf{Gran Sasso Quantum Meeting: From Many Particle Systems to Quantum Fluids}, GSSI L'Aquila}
\cvitem{Oct 2018}{\textbf{FRG Meeting: Long-Term Dynamics of Nonlinear Dispersive and Hyperbolic Equations}, University of Chicago}
\cvitem{May 2018}{\textbf{Conference on Nonlinear Waves}, Brown University}
\cvitem{May 2018}{\textbf{School and Conference on Nonlinear Waves: Stability vs Turbulence},  celebrating the contributions of Jalal Shatah, Georgia Tech}
\cvitem{Sept 2016}{\textbf{FRG Conference in Dispersive and Wave equations}, MIT}
\cvitem{July 2015}{\textbf{BCAM Workshop on Harmonic Analysis and PDEs}, BCAM}
\cvitem{July 2014}{\textbf{10th AIMS Conference on Dynamical Systems, Differential Equations and Applications}, ICMAT}
\cvitem{March 2014}{\textbf{IV School of Functional Analysis and Applications}, Brownian Motion and Ito's formula, Universidad de Sevilla}


%-------------------Teaching Experience------------------------------------------------------------------
\section{Teaching Experience}

\cvitem{Winter 2021}{\textbf{Math 316 - Differential Equations}, University of Michigan}
\cvitem{Fall 2020}{\textbf{Math 116 - Calculus II}, University of Michigan}
\cvitem{Spring 2020}{\textbf{Grader} for 18.615 - Introduction to Stochastic Processes, MIT}
\cvitem{Fall 2019}{\textbf{Grader} for 18.085 - Computational Science and Engineering I, MIT}
\cvitem{Spring 2019}{\textbf{Grader} for 18.615 - Introduction to Stochastic Processes, MIT}
\cvitem{Summer 2018}{\textbf{UROP+ Research Supervisor}, MIT}
\cvlistitem{\emph{Student:} Zixuan Xu}
\cvlistitem{\emph{Project:} Almost Conservation Laws for KdV and Cubic NLS} 
\cvitem{Spring 2018}{\textbf{Recitation Instructor} for 18.03 - Differential Equations, MIT}
\cvitem{Fall 2017}{\textbf{Recitation Instructor} for 18.02 - Multivariable Calculus, MIT}
\cvitem{Fall 2016}{\textbf{Grader} for 18.085 - Computational Science and Engineering I, MIT}
\cvitem{Summer 2016}{\textbf{UROP+ Research Supervisor}, MIT}
\cvlistitem{\emph{Student:} Eli Sadovnik}
\cvlistitem{\emph{Project:} A Central Limit Theorem for Fluctuations of Internal
Diffusion-Limited Aggregation with Multiple Sources} 

%-------------------Awards and Fellowships------------------------------------------------------------------

\section{Awards and Fellowships}

\cvitem{2015}{\textbf{Summer internship position}, Basque Center for Applied Mathematics (BCAM)}
\cvlistitem{ \emph{Advisor}: Luis Vega}
\cvlistitem{ \emph{Project:} Probabilistic interpretation of the Hardy uncertainty principle}
\cvitem{2014-2015}{\textbf{La Caixa Europe Fellowship}, La Caixa Foundation}
\cvlistitem{Full funding of master degree at the University of Cambridge}
\cvitem{2013-2014}{\textbf{Collaboration Scholarship}, Government of the Basque Country}
\cvlistitem{\emph{Advisor}: Luis Escauriaza}
\cvlistitem{ \emph{Project}: Harmonic Analysis and applications}
\cvitem{2012}{\textbf{Summer Research Scholarship}, University of Auckland}
\cvlistitem{\emph{Advisor}: Marston Conder}
\cvlistitem{\emph{Project}: Embeddings of circulant graphs}

%-------------------Languages Section-------------------------------------------------------------------

\section{Languages}

% Format:  \cvitemwithcomment{Language}{Skill level}{Comment}
% Example: \cvitemwithcomment{English}{Native}{Mother Tongue}
% Example: \cvitemwithcomment{French}{Fluent}{Daily practice, all work performed in English}

\cvitemwithcomment{}{\textbf{Basque}, Mother tongue}{Euskararen Gaitasun Agiria [C1], 2009}
\cvitemwithcomment{}{\textbf{Spanish}, Mother tongue}{}
\cvitemwithcomment{}{\textbf{English}, Fluent}{Certificate of Proficiency in English [C2], 2013}
\cvitemwithcomment{}{\textbf{Italian}, Good working knowledge}{}
\cvitemwithcomment{}{\textbf{French}, Intermediate}{French IV at MIT, 2020}
\cvitemwithcomment{}{\textbf{Portuguese}, Good working knowledge}{Portuguese I-IV at MIT, 2017-18}



%-------------------Cover letter------------------------------------------------------------------------

%\newpage

%\input{coverletter_UMass.tex}                             % Include cover letter from coverletter.tex


%-------------------Interests Section-------------------------------------------------------------------

%\section{Interests}

% Format:  \cvitem{Hobby}{Description}
% Example: \cvitem{Gaming}{Computer Games}
% Example: \cvitem{Sport}{Golf, Tennis}

%\cvitem{Gaming}{Computer Games}
%\cvitem{Sport}{Golf, Tennis}

%-------------------Experience Section------------------------------------------------------------------

%\section{Experience}


%-------------------Vocational Experience---------------------------------------------------------------

%\subsection{Vocational}

% Format: \tlcventry{StartYear}{EndYear}{Job title}{Employer}{City}{Country (optional)}{General description no longer than 1--2 lines.\newline{}%
% Example:
% \tlcventry{2008}{2011}{System Administrator}{Simple Solutions}{MyCity}{}{Did system administrative work.\newline{}%
% Main Duties:%
%  \begin{itemize}%
%      \item Administrate the servers;
%      \item Administrate employee computers 
%          \begin{itemize}%
%              \item All employee's computers had to be up to date;
%          \end{itemize}
%      \item Did some more administrating
%   \end{itemize}}

%\tlcventry{2008}{2011}{System Administrator}{Simple Solutions}{MyCity}{}{Did system administrative work.\newline{}%
%Main Duties:%
%\begin{itemize}%
% \item Administrate the servers;
% \item Administrate employee computers 
%  \begin{itemize}%
%      \item All employee's computers had to be up to date;
%      \end{itemize}
%  \item Did some more administrating
%\end{itemize}}

%-------------------Skills Matrix Section----------------------------------------------------------------

%\section{Skills}

% For items with categories: 
% Format:  \cvdoubleitem{Category}{List of skills}{Category Name}{List of skills}
% Note: It looks better if the category is bold with \textbf{}
% Example:
% \subsection{Development}
% \cvdoubleitem{\textbf{Languages}}{C\#, C\+\+, Java}{\textbf{Databases}}{MSSQL, MySQL}
%
% For a bullet list without categories:
% Format:  \cvlistdoubleitem{Skill 1}{Skill 2}
% Example: 
% \subsection{Development}
% \cvlistdoubleitem{C\#, Java, Ruby}{MSSQL, MySQL}
% \cvlistdoubleitem{Photoshop}{Windows, Linux. In the single column list, this item is particularly long to wrap over several lines.}

%\subsection{Development}
%\cvdoubleitem{\textbf{Languages}}{C\#, Java, Ruby}{\textbf{Databases}}{MSSQL, MySQL}


%-------------------Publications Section----------------------------------------------------------------
% The cvitem commands needs to be altered to correctly print all publications with the moderntime package.
% The cvitem command is edited to remove all forced punctuation within the command.
% All the typesetting of the text is handled by the modified Biblatex style.

%\input{cvitem_modifications/cvitem_modified}        % Removing forced punctuation from cvitem

%\nocite{*}                                          % Print all publications.

% Format:  \printbibliography[type=Biblatex type,title={Title of publication}]
% Example: \printbibliography[type=article,title={Journal Publications}]
% Example: \printbibliography[type=inproceedings,title={Conference Publications}]
% Example: \printbibliography[type=thesis,title={Thesis}]

%\printbibliography[type=article,title={Journal Publications}]
%\printbibliography[type=inproceedings,title={Conference Publications}]
%\printbibliography[type=thesis,title={Thesis}]

%\input{cvitem_modifications/cvitem_moderncvclassic} % Reverting changes to cvitem.

%-------------------References Section------------------------------------------------------------------

%\section{References}

% Format:  \cvreferencecolumn{\cvreference{Name Surname}{Position}{Department}{Company}{City}{Email}{Home Phone}{Cell Phone}}{\cvreference{Name Surname}{Position}{Department}{Company}{City}{Email}{Home Phone}{Cell Phone}}
% Example: 
% \subsection{Simple Solutions}
% \cvreferencecolumn{\cvreference{John Doe}{Developer}{HR}{Simple Solutions}{MyCity}{john@email.com}{+12 (34) 567 8901}{+23 (45) 678 9012}}{\cvreference{Jane Doe}{Accountant}{HR}{Simple Solutions}{MyCity}{jane@email.com}{+34 (56) 789 0123}{+45 (67) 890 1234}}
% \subsection{Monster Inc}
% \cvreferencecolumn{\cvreference{Alice Doe}{Manager}{HR}{Monster Inc}{ThatCity}{alice@email.com}{+12 (34) 567 8901}{+23 (45) 678 9012}}{}

%\subsection{Simple Solutions}
%\cvreferencecolumn{\cvreference{John Doe}{Developer}{HR}{Simple Solutions}{MyCity}{john@email.com}{+12 (34) 567 8901}{+23 (45) 678 9012}}{\cvreference{Jane Doe}{Accountant}{HR}{Simple Solutions}{MyCity}{jane@email.com}{+34 (56) 789 0123}{+45 (67) 890 1234}} \subsection{Monster Inc}
%\cvreferencecolumn{\cvreference{Alice Doe}{Manager}{HR}{Monster Inc}{ThatCity}{alice@email.com}{+12 (34) 567 8901}{+23 (45) 678 9012}}{}
%
%\clearpage

%-------------------Appendix----------------------------------------------------------------------------
% This section is added to append any additional documents to the cv.
% The appended documents are added to the table of contents for easier navigation of the document.
% Usage: (section)
% \phantomsection
% \addcontentsline{toc}{section}{title}
% 
% Format: (subsection)
% \phantomsection\addcontentsline{toc}{subsection}{title}
% \includepdf[pages=-]{appendix/filename.pdf}
%
% Example:
% \phantomsection
% \addcontentsline{toc}{section}{Certificates}
%
% \phantomsection
% \addcontentsline{toc}{subsection}{Landscape}
% \includepdf[pages=-]{appendix/CertificateLandscape.pdf}
%
% \phantomsection
% \addcontentsline{toc}{subsection}{Portrait}
% \includepdf[pages=-]{appendix/CertificatePortrait.pdf}

%\phantomsection
%\addcontentsline{toc}{section}{Certificates}
%
%\phantomsection
%\addcontentsline{toc}{subsection}{Landscape}
%\includepdf[pages=-]{appendix/CertificateLandscape.pdf}
%
%\phantomsection
%\addcontentsline{toc}{subsection}{Portrait}
%\includepdf[pages=-]{appendix/CertificatePortrait.pdf}





\end{document}

%% end of file `main.tex'.
