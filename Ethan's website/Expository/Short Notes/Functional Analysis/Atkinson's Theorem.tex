\documentclass[12pt]{article}

\usepackage{/Users/ethanjaffe/Documents/myMacros}
\usepackage{/Users/ethanjaffe/Documents/mySettings}

\title{Atkinson's Theorem}
\author{Ethan Y. Jaffe}
\date{}

\begin{document}
\maketitle
\setcounter{section}{1}
In this note we give a proof of a version of Atkinson's Theorem for Fredholm operators on Banach spaces.

Recall that if $V,W$ are Banach spaces, a bounded operator $T:V\to W$ is called Fredholm if $\ker T$ and $\cok T = W/\im T$ are finite dimensional, and $\im T$ is closed. In fact, $\cok T$ finite-dimensional implies that $\im T$ is closed (see Lemma~\ref{op}). If $V=W=H$ is a Hilbert space, then $T:H\to H$ is Fredholm if $\ker T$ is closed, $\im T^\bot$ is closed, and $\im T$ is finite-dimensional. None of these conditions is redundant since we need to assume that $\im T$ is closed to have an isomorphism $\cok T \to \im T^\bot$.

We will also denote by $U'$ the continuous dual space to the normed vector space $U$.

We recall Atkinson's Theorem which says:
\begin{thm}[Atkinson]A bounded operator on a Hilbert space $T:H \to H$ is Fredholm if and only if there exist $R,S$ such that $I-RT$ and $I-TS$ are compact.\end{thm}
\begin{rk}We remark that we can always assume $R=S$ in the above. Indeed, if $RT = I-K_1$ and $TS = I-K_2$, then notice that
\[S-K_1S = (I-K_1)S = (RT)S = R(TS) = R(I-K_2) = R-RK_2,\]
and so $S-R = K_3$ is compact, and hence
\[ST = RT + K_3T = I -I_1+K_3T,\] where the sum of the latter two operators is compact, and similarly for $TR$.\end{rk}
We give a quick proof.
\begin{proof}Suppose $R,S$ exist as above. If $\ker T$ were infinite-dimensional, we could extract an infinite orthonormal sequence $v_n \in \ker T$. Then $RTv_n = 0 = v_n-K_1v_n$. Passing to a subsequence of $K_1v_n$, we see that $v_{n_k}$ is convergent, which contradicts orthogonality. Observe that $\im T^\bot = \ker T^\ast$, and that $S^\ast T^\ast = (TS)^\ast = I-K_2^\ast$. $K_2^\ast$ is compact since it is also the limit of finite-rank operators (an alternative, more general, proof is given below in Lemma~\ref{comp}). Thus we apply the previous argument to $T^\ast$ to conclude that $\im T^\bot$ is finite-dimensional.

Next we show that $\im T$ is closed. Suppose $v_n = Tw_n \to v$. Without loss of generality, we may assume that $w_n \in \ker T^\bot$. Then $Rv_n = RTw_n = w_n -K_1w_n \to Rv$. First assume $\sup_n \norm{w_n} = \infty$. Then passing to a subsequence,  we may assume $Tw_n/\norm{w_n} \to 0$, and hence $RTw_n/\norm{w_n} = w_n/\norm{w_n} - K_1w_n/\norm{w_n} \to 0$. Since $K_1$ is compact, passing to a futher subsequence, we obtain that $w_n/\norm{w_n}$ is convergent to some $u$, with $\norm{u} = 1$. Thus $Tu = \lim Tw_n/\norm{w_n} = 0$. But $u \in \ker T^\bot$, and so $u = 0$, too, a contradiction. Thus the $w_n$ are unifomrly bounded. Then $w_n - K_1w_n = Rv_n \to Rv$, and so passing to a subsequence and using that $K_1$ is compact, $w_n \to w$, and $Tw = \lim Tw_n = v$, and so $v \in \im T$.

Conversely, if $T$ is Fredholm, then $T|_{\ker T^\bot}:\ker T^\bot \to \im T$ is invertible. Let $R$ be its inverse, which is bounded by the open mapping theorem since $\im T$ is closed and hence a Hilbert space. Since $\im T$ is closed, $H = \im T\oplus \im T^\bot$, and thus we may extend $R$ to a map on all of $H$ by setting $R|_{\im T^\bot} \equiv 0$. Considering the cases $v \in \ker T$ and $v \in \ker T^\bot$ separately, it is clear that $RTv = I-P_{\ker T}$. Here, $P_V$ denotes the projection onto a subspace $V$, which is finite rank (and hence compact) if $V$ is finite-dimensional, as is in the given case. Similarly, consider $v \in \im T$ and $v \in \im T^\bot$ seperately, $TR = I-P_{\im T^\bot}$. This completes the proof.\end{proof}

We now state the generalization to Banach spaces.
\begin{thm}[Banach Space Atkinson]A bounded operator between Banach spaces $T:V \to W$ is Fredholm if and only if there exist $R,S$ such that $I-RT$ and $I-TS$ are compact.\end{thm}
For Banach spaces, an operator is compact if the image of the unit ball is precompact, i.e. every bounded sequence gets mapped to one which a convergent subsequence.

\begin{proof}The idea is to mimic the proof for Hilbert spaces. We will mention the subsitutes to constructions used in Hilbert spaces and prove them as lemmas after the proof.

First suppose that $I-RT$ and $I-TS$ are compact. To prove that $\ker T$ is finite-dimensional we need only find a substitute for orthonormality. This is given by the so-called Riesz lemma (Lemma~\ref{riesz}), a corollary of which asserts that if $\ker T$ is infinite-dimensional, there exists a sequence $v_n$ with $\norm{v_n} = 1$ and $\norm{v_n-v_m} \geq 1/2$ for any $n\neq m$. In particular no subsequence of $v_n$ is convergent, and the argument above goes through.

We now need to prove that $\cok T$ is finite-dimensional, which also implies that $\im T$ is closed by Lemma~\ref{op}. By Lemma~\ref{dual}, it suffices to show that $(\cok T)'$ is finite-dimensional.
Denote (suggestively) by $\im T^\bot \subseteq W'$ the set of all functionals vanishing on $\im T$. Ignoring the topology for now, we should have an isomorphism $\im T^\bot \leftrightarrow (\cok T)'$, since any functional vanishing on $\im T$ induces one on $\cok T$ and vice-versa. Lemma~\ref{bounded} asserts that this is in fact an isometry of normed spaces (with the operator nom).

So we need to prove that $\im T^\bot$ is finite-dimensional. We need a substitute for adjoints on Hilbert spaces. Fortunately, these exist: if $A:X\to Y$, we may define its adjoint $A^\ast: Y' \to X'$ by $A^\ast(\phi)= \phi\circ A$. It is easy to check that $A^\ast$ is bounded if $A$ is. It is also clear that $\im T^\bot = \ker T^\ast$, that $I^\ast = I$ and that $(AB)^\ast = B^\ast A^\ast$ if $B:Z\to X$ and $A:Y \to Y$. What is slightly harder to check is that if $K:X\to Y$ is compact, then $K^\ast:Y' \to X'$ is compact. This is the content of Lemma~\ref{comp}.

With this, we observe as in the case for Hilbert spaces that $T^\ast$ satisfies the hypotheses of Atkinson's theorem, and so $\ker T^\ast = \im T^\bot \iso \cok T$ is finite-dimensional.

To prove the converse, we need a substitute for orthogonal complements. Unfortunately these don't exist generically for even closed subspaces of Banach spaces, but they do for the cases that we need. Lemma~\ref{compl} provides for this. Given this, we may write $V = \ker T + X$, where $X \n \ker T = 0$, $X$ is closed, and $W = \im T + Y$, where $Y\n \im T = 0$ and $Y$ is finite-dimensional and closed, and furthermore the projections onto each summand (which are well-defined because each $v \in V$ and $w \in W$ decomposes uniquely) are bounded. With this set up, the proof of the converse for Hilbert spaces goes through nearly verbatim. \end{proof}

\section{Lemmas}
We now prove the various lemmas used in the proof.

\begin{lem}\label{op}Suppose $V,W$ are Banach spaces, and $T:V \to W$ is bounded. If $\dim \cok T < \infty$, then $\im T$ is closed.\end{lem}
\begin{proof}
Quotienting out $V$ by $\ker T$, and replacing $T$ by the map on the quotient space (which is still bounded by Lemma~\ref{bounded}) does not change $\im T$, but now we may consider $T$ to be injective.

Pick a basis $\{w_i+\im T\}$ of $\cok T = W/\im T$. Rescaling, we may assume that $\norm{w_i} = 1$. Since $\cok T \iso \C^d$ for some $d < \infty$, putting $\norm{\sum a_i(w_i+\im T)}' = \sup |a_i|$ gives a norm on $\cok T$, which is equivalent to the quotient norm on $\cok T$.
Define a norm on $\C^d \oplus V$ by
\[\norm{(x,v)} = \sup_{i} |x_i| + \norm{u}.\]
Notice that this norm makes $\C^d \oplus V$ into a Banach space since both $\C^d$ and $V$ are, and convergence with respect to the norm on $\C^d \oplus V$ is just convergence with respect both convergence in $\C^d$ and $V$.

We define a map
\[\C^d\oplus V \to W\] by
\[(a_1,\ldots,a_d,u) \mapsto a_1w_1+\cdots + a_dw_d + Tv.\]
We will show that this map is a bounded linear bijection, and hence an isomorphism by the open mapping theorem. Since $V$ is certainly a closed subspace of $\C^d\oplus V$, its image under the isomorphism, which is precisely $\im T$, is closed in $W$, which is what we need to show.

This map is clearly linear. It is also injective since $a_1w_1+\cdots a_dw_d + Tv = 0$ means that $a_1w_1+\cdots a_dw_d \in \im T$, which means that 
\[a_1(w_1+\im T)+\cdots a_d(w_d+\im T) = 0\] which means that all $a_i = 0$. We can thus conclude that $Tv = 0$, which means that $v=0$. It is also surjective since if $w\in W$, then
\[w+\im T = a_1(w_1+U)+\cdots a_d(w_d+U)\] for some $a_i$, in which case we may set
\[w'= w-(a_1w_1+\cdots a_dw_d) \in \im T,\]
and hence $w' = Tv$ for some $v \in V$.

Lastly, we show boundedness. This is easy since
\[\norm{a_1w_1+\cdots a_dw_d + Tv} \leq d\sup |a_i| + \norm{Tv} \leq (d+\norm{T})\norm{(a_1,\ldots,a_d,v)}.\]\end{proof}
\begin{rk}It may seem as though considering $T$ is redundant. However, if $U \subseteq V$ is a subspace of a Banach space, and $V/U$ is finite-dimensional, then it's not necessarily true that $U$ is closed. Indeed, consider $\ell^2(\C)$. Extend any orthonormal basis to a vector space basis by adding on a non-empty (in fact uncountable) collection of vectors $\{v_{\alpha}\: \alpha \in A\}$. Let $U$ be the subspace spanned by all the orthonormal basis vectors and all but one of the $v_\alpha$. Then $\dim \ell^2(\C)/U = 1$, but $U$ is dense in $\ell^2(\C)$, and thus is in particular not closed.\end{rk} 

\begin{lem}{\label{comp}}Suppose $V,W$ are Banach spaces and $K:V\to W$ is compact. Then $K^\ast:W' \to V'$ is compact as well.\end{lem}
\begin{proof}
Suppose $\phi_n \in W'$ are uniformly bounded in the operator norm. Then by definition $K^\ast(\phi_n) = \phi_n\circ K$, and so we need to show that $\phi_n\circ K$ has a uniformly Cauchy subsequence.

By compactness of $K$, $L = \overline{K(\overline{B(0,1)})}$ is compact in $W$. Since the $\phi_n$ are uniformly bounded in the operator norm and are linear, they are equicontinuous. In particular by Arzela-Ascoli, and passing to a subsequence without changing notation, we may assume that $\phi_n|_{L}$ are uniformly Cauchy, i.e. for all $\epsilon > 0$ if $m,n$ are large enough, then for all $y \in L$,
\[|\phi_n(y)-\phi_m(y)| < \epsilon.\] This means that for all $x \in V$, $K(x/|x|) \in L$ and so
\[|\phi_n(K(x))-\phi_m(K(x))| < |x|\epsilon,\]
i.e. the family of functionals $\phi_n\circ K$ are Cauchy in the operator norm, which is what we needed to show.
\end{proof}
\begin{lem}[Riesz]\label{riesz}Suppose $V$ is a Banach space and $U \subsetneq V$ is a closed proper subspace. For $v \in V$ set $\norm{v+U} = \inf_{u \in U} \norm{v-u}$. Then for all $\epsilon > 0$ there exists $v \in V$ with $\norm{v} = 1$ and $\norm{v+U} \geq 1-\epsilon$.\end{lem}
As a corollary we get the substitute for orthonormality
\begin{cor}\label{subs}If $V$ is an infinite-dimensional Banach space, then there exists a sequence in $v_n$ with $\norm{v_n} = 1$ and $\norm{v_n-v_m} \geq 1/2$ whenever $v \neq m$.\end{cor}
\begin{proof}[Proof of Lemma~\ref{riesz}]We try the only thing we can. We know since $U$ is closed that there exists $x \in V$ with $\norm{x+U} = \rho > 0$. We only have access to $x$ and $U$, so the only thing we can do is take linear combinations of $x$ and things in $U$. Since $U$ is invariant under dilation, all things are of the form
\[v = \frac{x-u}{\lambda}.\] If we want $\norm{v} = 1$, we must have $\lambda = \norm{x-u} \geq \rho > 0$. We also want $\norm{(x-u)\lambda^{-1}+U} = \lambda^{-1}\norm{x+U} = \lambda^{-1}\rho\geq 1-\epsilon$. The only way this can happen is if $\lambda \leq \rho(1-\epsilon)^{-1}$. Since $(1-\epsilon)^{-1} > 1$, there is enough room to find some $u \in U$ with $\lambda = |x-u| \leq (1-\epsilon)^{-1}\rho$.\end{proof}
\begin{proof}[Proof of Corollary~\ref{subs}]Let $v_1 \in V$, $\norm{v_1} = 1$, and set $V_1 = \vspan(v_1)$. Then $V_1$ is finite-dimensional, and thus closed, and so there exists $v_2 \in V$, $\norm{v_2} = 1$ and $\norm{v_2+V_1} \geq 1/2$. Set $V_2 = \vspan(v_1,v_2)$. Then we can find $v_3 \in V$ with $\norm{v_3} = 1$ and $\norm{v_3+V_2} \geq 1/2$. Continue in this fashion to define $v_n$. Then if $n < m$ $\norm{v_n-v_m} \geq \norm{v_m+V_n} \geq 1/2$.\end{proof}

\begin{lem}\label{dual}Suppose $V$ is a Banach space. If $V'$ is finite-dimensional, so is $V$.\end{lem}
\begin{proof}If $V$ were reflexive, we'd be done since it is clear that $V'' \iso V$ is finite-dimensional. The theorem is important since it is true even without this assumption.

Suppose $V$ were infinite-dimensional. Let $\{v_1,v_2,v_3,\ldots\}$ be an infinite linearly-independent set, and set $V_i = \vspan(v_1,\ldots,v_i)$. Define a functional $\psi_i:V_i \to \C$ by $\psi_i(v_j) = \delta_{ij}$. Since $V_i$ is finite-dimensional, $\psi_i$ is bounded and so extends by the Hahn-Banach theorem to a functional $\psi_i \in V'$. Now the collection of the $\psi_i$ are linearly independent. Indeed, if 
\[a_1\psi_1+\cdots + a_n\psi_n = 0,\] then we may plug in $v_i$, $1 \leq i \leq n$ to see that all $a_i = 0$. This contradicts that $V'$ is finite-dimensional.\end{proof}
\begin{rk}The lemma is still true, with a similar proof, if one merely assumes that $V$ is a Hausdorff locally-convex vector space. The lemma is not true in general, even with the assumption of Hausdorff. For example, the space $L^{1/2}([0,1]) = \{\text{measurable }f:[0,1] \to \C\: \int_0^1 |f|^{1/2} < \infty\}$ endowed with the metric $d(f,g) = \int_0^1 |f-g|^{1/2}$ is certainly infinite dimensional (and is also a complete metric space, so is in particular Hausdorff), but its dual is the trivial vector space.\end{rk}

\begin{lem}\label{bounded}Suppose $V$ is a Banach space, and $U \subseteq V$ is a closed subspace. Let $W$ be any normed vector space. Then there is an isometry between the space of bounded operators $T:V\to W$ which vanish on $U$ and the space of bounded operators $T:V/U \to W$ (where we give the domain the quotient norm) defined by sending $T:V \to W$ vanishing on $U$ to
\[\overline{T}:V/U \to W\] defined by $\overline{T}(v+U) = T(v)$.\end{lem}
\begin{proof}If $T:V \to W$ vanishes on $U$, then $\overline{T}$ is clearly well-defined. We will show that if $T$ is bounded, then so is $\overline{T}$. In fact we will show that the assingment $T \mapsto \overline{T}$ is norm-decreasing. Given $v+U \in V/U$, we may pick $v_n \in v+U$ so that $\norm{v_n} \to \norm{v+U}$. Also for all $n$,
\[\norm{\overline{T}(v+U)} = \norm{\overline{T}(v_n+U)} = \norm{T(v_n)} \leq \norm{T}\norm{v_n}.\] Taking limits shows that $\norm{\overline{T}(v+U)} \leq \norm{T}\norm{v+U}$.

Conversely, if $\overline{T}:V/U \to W$ is bounded, we may define $T(v) = \overline{T}(v+U)$. This turns $T$ into a map $V \to W$ which vanishes on $U$. We now show that $T$ is bounded and the map $\overline{T} \mapsto T$ is norm-decreasing. Indeed,
\[\norm{T(v)} = \norm{\overline{T}(v+U)} \leq \norm{\overline{T}}\norm{v+U} \leq \norm{\overline{T}}\norm{v}.\]

It is clear that the above assignments are inverses of each other. Since both assignments are norm-decreasing, it follows that they are both in fact isometries.\end{proof}

\begin{lem}\label{compl}Suppose $V$ is a Banach space. If $X \subseteq V$ satisfies
\begin{enumerate}[label = (\roman*)]
\item  $X \subseteq V$ is finite-dimensional, or
\item $X \subseteq V$ is closed and $V/X$ is finite-dimensional
\end{enumerate}
then there exists a complementary subspace $Y \subseteq V$ closed with $V = X+Y$ and $X\n Y = 0$, and the projections $P_X$ and $P_Y$ are well-defined and bounded. Furthrmore, in the case of (ii), $\dim Y  = \dim V/X$. Here the projections $P_X$ are defined by $P_X(x+y) = x$ where $x+y$ is the unique such decomposition of an element in $V$, and similarly for $P_Y$.\end{lem}
\begin{proof}To show the existence of $Y$ and the boundedness of the projections, it suffices to find a bounded operator $P_X:V \to X$ which is the identity on $X$. Indeed, we will show that we can take $Y = \ker P_X$.

For any $v \in V$, $v = P_X v + (1-P_X)v$. The first term is in $X$. Since $P_X$ is the identity on $X$, $P_X^2 = P_X$,and so the second term is in $\ker P_X = Y$. This shows that $X + Y = V$. If $v \in X\n \ker P_X$, then $v = P_Xv = 0$, so $X\n \ker P_X = 0$. By definition, $v = P_Xv + (1-P_X)v$, so $P_X$ is the projection onto $X$, and is bounded by assumption. Since $P_Y = 1-P_X$, it is also bounded.

We now show the existence of $P_X$ in the case of (i) and (ii). Suppose $X \subseteq V$ is finite-dimensional, and pick a basis $x_1,\ldots, x_n$ of $X$. Then any $x \in X$ has a unique representation 
\[x = \sum a_ix_i,\] for some $a_i \in \C$. The assignment $x \mapsto a_i$ is actually a linear functional on $X$,which we henceforth denote by $a_i(x)$. Since $\dim X< \infty$, $a_i$ is bounded. Thus $a_i$ extends to a bounded functional on all of $V$. This lets us define a bounded $P_X:V \to X$ by 
\[P_X(v) = \sum a_i(v)x_i,\]
which is by definition the identity on $X$.

Now suppose $X \subseteq V$ is closed and $\dim V/X$ is finite. Pick $v_i \in V$ so that the collection $\{v_i+X\}$ form a basis of $V/X$. Since $V/X$ is finite-dimensional, it is isomorphic (but perhaps not isometric) as a Banach space to $\C^d$ for some $d < \infty$, where we put the $\sup$ norm on $\C^d$, with the isomorphism taking
\[a_1(v_1+X)+\cdots a_d(v_d+X) \mapsto (a_1,\ldots,a_d).\]
Similar to the proof in Lemma~\ref{op}, we define a map
\[T: \C^d\oplus X \to V\]  by
\[(a_1,\ldots,a_d,x) \mapsto a_1v_1 + \cdots a_dv_d + x.\]
It is clear (like in the proof of Lemma~\ref{op}) that this map is surjective, injective, and bounded. Likewise, since $X$ is a closed subspace of a Banach space, it is a Banach space itself, and thus $\C^d \oplus X$ is a Banach space under any obvious norm. The image of $0\oplus X$ under $T$ is clearly just $X \subseteq V$ itself. Let $P:\C^d \oplus X \to 0\oplus X$ be the projection onto the second factor, which is bounded. Define the bounded operator $P_X = TPT^{-1}:V \to V$, which certainly maps $V$ into $X$, and is the identity on $X$. 

Thus we have shown in both cases (i) and (ii) the existence of the complementary subspace $Y$ and the boundedness of the projections. In the case of (ii), we need to show that $\dim Y = \dim V/X$.

It is clear that $\ker P_Y = X$. By Lemma~\ref{bounded}, $P_Y:V \to Y$ descends to a map $\overline{P_Y}:V/X \to Y$, which is surjective. It is injective, for if $0 = \overline{P_Y}(v+X) = P_Yv$, then $v \in \ker P_Y = X$, and so $v+X = 0$. Thus $\overline{P_Y}$ is a linear isomorphism, and $\dim Y = \dim V/X$.
\end{proof}
\end{document}
