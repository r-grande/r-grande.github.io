\documentclass[12pt]{article}

\usepackage{/Users/ethanjaffe/Documents/Work/myMacros}
\usepackage{/Users/ethanjaffe/Documents/Work/mySettings}
\usepackage{appendix}

\title{Fredholm Properties of Elliptic Peusodifferential Operators}
\author{Ethan Y. Jaffe}
\date{}

\begin{document}
\setcounter{section}{1}
\maketitle
Let $M$ be a compact manifold without boundary, and $P$ an elliptic pseudodifferential operator of order $s$. Then $P:H^{t+s}(M) \to H^{t}(M)$ is bounded for all $t \in \R$. We will be pedantic write $P_{t+s \to s}$ to denote $P$ with domain $H^{t+s}$. There is also another elliptic pseudodifferential operator $Q$ of order $-s$ such that $PQ \equiv QP \equiv I$ modulo smoothing operators. Let $P^\ast$ denote the formal adjoint of $P$, which is again an elliptic pseudodifferential operator of order $s$. By elliptic regularity, $\ker P_{t+s \to t} \subseteq C^\infty(M)$ does not depend on $t$, and likewise for $P^\ast$. We will call this common space $\ker P$ (resp. $\ker P^\ast$).

We show the following Fredholm properties of $P$:
\begin{thm}\label{one}Let $s \in \R$. Then we have the decompositions
\[H^t(M) = \ker P \oplus \im (P^\ast)_{t+s\to t} = \ker P^\ast \oplus \im P_{t+s\to t}\]
where the sums are orthogonal with respect to the $L^2$ inner product (in the case that $t < 0$, one must interpret this as a distributional paring on $C^\infty(M)$).

Also, $\dim \ker P, \dim \ker P^\ast < \infty$. Then, $L^2$ projections $\Pi_V$ ($V = \ker P,\ker P^\ast$) exist and are smoothing operators. Furthermore, there exists a pseudodifferential $\tilde{Q}$ of order $-s$ such that
\[\tilde{Q}P = I-\Pi_{\ker P}, \ \ P\tilde{Q} = I-\Pi_{\ker P^\ast}.\] $\tilde{Q}$ is formally self-adjoint if $P$ is.

Lastly, we also have the analogous decompositions
\[C^\infty(M) = \ker P \oplus P^\ast(C^\infty(M)) = \ker P^\ast \oplus P(C^\infty(M)),\]
and
\[\mathcal D'(M) = \ker P \oplus P^\ast(\mathcal D'(M)) = \ker P^\ast \oplus P(\mathcal D'(M)),\]
which are also orthogonal (the second of the pair of decompositions is again interpreted via the distributional pairing). \end{thm}

We will use this theorem to prove the following well-known result:
\begin{thm}\label{two}Let $P$ be a formally self-adjoint elliptic pseudodifferential operator of order $s \neq 0$ acting on a compact manifold $M$. Then there exist a sequence of smooth functions $u_n$ and real numbers $\lambda_n$ such that $Pu_n = \lambda_n u_n$ and the $u_n$ (suitably normalized) form an orthonormal basis of $L^2(M)$. If $s > 0$, $|\lambda_n| \to \infty$. if $s < 0$, $\lambda_n \to 0$. In particular $P$ is an essentially self-adjoint (possibly unbounded) operator on $L^2(M)$. If $s = 0$, then $P$ is still self-adjoint\end{thm}
In particular for the case $-\Delta + 1$ this is true. Since $-\Delta+1$ is a formally positive operator with no kernel, all of its eigenvalues are $> 0$. There is no conclusion about eigenvalues for the case, $s=0$ as the identity operators shows.

\begin{proof}[Proof of Theorem~\ref{one}]Smoothing operators are bounded between all $H^t(M)$, and in particular are compact on all $H^t(M)$. The Fredholm theorem then applies, and so $\ker P$, $\cok P_{t \to t+s}$ are finite-dimensional, $\im P_{t \to t-s}$ is closed, and similarly for $P^\ast$. Notice now that $(P^\ast)_{s-t\to -t} = (P_{t \to t-s})^\ast$, i.e. the Banach space adjoint of $P_{t \to t+s}$ is given by $P^\ast$> Indeed, the operators both satisfy
\[(P^\ast u|v) = (u|Pv) = (P_{t \to t-s}^\ast u|v)\] for all $u,v \in C^\infty$, and so
\[P^\ast u = (P_{t \to t-s})^\ast u\] for all $u \in C^\infty$. Since both operators bounded on $H^{s-t}$, they are the same. Thus we will just write $P^\ast_{-t+s\to-t}$ for both operators.

We therefore have the following decomposition:
\[L^2(M) = \ker P \oplus  \ker P^\bot.\] It is clear that $\ker P = (\im P^\ast_{s \to 0})^\bot,$ and since $\im P^\ast_{s \to 0}$ is closed, that $\ker P^\bot = \im P^\ast_{s \to 0}$. 

Now fix $t > 0$. If $u \in H^t(M),$, then $u \in L^2(M)$ and so $u = v + w$ for $v \in \ker P$, $w \in \im P^\ast_{s\to 0}$, and $(v|w) = 0$. Since $v \in C^\infty(M) \subseteq H^t(M)$, $w \in H^t(M)$ as well, and thus by elliptic regularity, $w = P^\ast w'$ for some $w' \in s+t$. This proves that
\[H^t(M) = \ker P \oplus \im P^\ast_{t+s\to t},\] and the decomposition is orthogonal. The same argument proves the $C^\infty(M)$ decomposition.

Now fix $t < 0$. Then by density
\[H^t(M) = \overline{\ker P \oplus \im P^\ast_{s\to 0}}.\] Now $\ker P$ is finite-dimensional, and so $\ker P \oplus\overline{ \im P^\ast_{s\to 0}}$ is closed. So
\[H^t(M) = \ker P \oplus\overline{\im P^\ast_{s\to 0}}.\] By continuity, this decomposition is orthogonal. Since $\im P^\ast_{t+s\to t}$ is closed and contains $\im P^\ast_{s \to 0}$,
\[\overline{\im P^\ast_{s\to 0}} \subseteq \im P^\ast_{t+s\to t}.\] Conversely, if $u = P^\ast v$ where $v \in H^{t+s}(M)$, then $v = \lim v_n$ where $v_n \in H^{s}(M)$, and so $u = \lim Pv_n \in \overline{\im P^\ast_{s\to 0}}$. So we have the decomposition
\[H^t(M) = \ker P \oplus \im P^\ast_{t+s\to t}.\]

The decomposition for $\mathcal D'(M)$ follows immediately from the above and the fact that
\[\mathcal D'(M) = \bigcup_{s \in \R} H^s(M).\]

Since $P^{\ast\ast} = P$, we have the other of the two decompositions, too.

If $V \subseteq C^\infty(M)$ is finite-dimensional, we may pick a basis $e_1,\ldots, e_n$ of $V$. Then $\Pi_V$ has kernel $\sum_{i=1}^n \overline{e_i(y)}e_i(x)$ and is thus smoothing. Define $\tilde{Q}$ first on $C^\infty(M)$ as follows: by the decomposition, $P$ is injective on $P^\ast(C^\infty(M))$. So we may define $\tilde{Q}:P(C^\infty(M)) \to P^\ast(C^\infty(M))$ to be its two-sided inverse. Extend $\tilde{Q}$ to an operator on all of $C^\infty(M)$ by setting $\tilde{Q}(\ker P^\ast) = 0$. By the decomposition, this makes $\tilde{Q}$ well-defined. Notice that the decomposition
\[v = \Pi_{\ker P^\ast}v + (I-\Pi_{\ker P^\ast})v\] is an orthogonal decomposition, and so $(I-\Pi_{\ker P}v) \in P(C^\infty(M))$. It follows that 
\[P\tilde{Q}v = P\tilde{Q}(I-\Pi_{\ker P^\ast})v = (I-\Pi_{\ker P^\ast})v.\] Similiar, we may write
\[v = \Pi_{\ker P}v + (I-\Pi_{\ker P})v,\] and so
\[\tilde{Q}Pv = \tilde{Q}P(I-\Pi_{\ker P})v = (I-\Pi_{\ker P})v.\]

We next extend $\tilde{Q}$ to a bounded operator $H^{t}(M) \to H^{t+s}(M)$ for all $t \in \R$. This is done in a similar way to the above. $P$ is injective on $\im (P^\ast_{t+s\to t}$to $\im P_{t+s\to t}$ and so has an inverse $\tilde{Q}$. This $\tilde{Q}$ agrees with the one obtained above on $P^\ast(C^\infty(M))$ since we have the inclusion $P^\ast(C^\infty(M)) \subseteq \im (P^\ast_{t+s\to t}$. Since $\im P_{t+s \to t}$ is closed, by the open mapping theorem, $\tilde{Q}$ is bounded. We also define, as above, $\tilde{Q}$ to be $0$ on $\ker P$. This definition extends $\tilde{Q}$. $\tilde{Q}$ is bounded restricted to $\ker P$ and to $\im P_{t+s\to s}$. This implies that it is bounded on their sum, which is $H^{t}(M)$. See Lemma~\ref{sdsd}. Since $\Pi_{V}$ is already bounded on all $H^{t}$ spaces ($V = \ker P, \ker P^\ast$), $\tilde{Q}Pv = I-\Pi_{\ker P}$ and $P\tilde{Q} = (I-\Pi_{\ker P^\ast})$.

From this, one shows easily using Sobolev embedding that $\tilde{Q}$ is continuous on $C^\infty(M)$.

Next we show that $\tilde{Q}$ is pseudodifferential. Indeed, if $Q$ is any pseudodifferential parametrix to $P$ then $PQ = I-K$, where $K$ is smoothing. Thus,
\[\tilde{Q}PQ = \tilde{Q}(I-K) = \tilde{Q}-\tilde{Q}K.\] On the other hand,
\[\tilde{Q}PQ = (I-\Pi_{\ker P})Q = Q-Q\Pi_{\ker P}.\] Now $Q\Pi_{\ker P}$ is smoothing, since $Q$ is continuous on $C^\infty(M)$. Similarly, $\tilde{Q}K$ is smoothing. Thus $\tilde{Q}-Q$ is smoothing. Since $Q$ is pseudodifferential, so is $\tilde{Q}$.

Finally we show that $\tilde{Q}$ is formally self-adjoint if $P$ is. If $u,v \in C^\infty(M)$, we write $u = u'+u''$ and $v = v'+v''$ where $u',v' \in \ker P$ and $u'', v'' \in P(C^\infty(M))$. Since $P^\ast = P$, by definition this means that $\tilde{Q}u = \tilde{Q}u''$ and $\tilde{Q}v = \tilde{Q}v''$. Write $u'' = Px$ and $v'' = Py$. Then
\[(\tilde{Q}u|v) = (\tilde{Q}u''|v) = (\tilde{Q}Px|v' + Py) = ((I-\Pi_{\ker P})x|v') + ((I-\Pi_{\ker P})x|Py).\]
Since $v' \in \ker P$, the first sum is $0$. Since $x$ is orthogonal to $\ker P$, $(I-\Pi_{\ker P})x = x$. So
\[(\tilde{Q}u|v) = (x|Py) = (Px|y).\] Arguing in reverse shows formal self-adjointness.\end{proof}

\begin{proof}[Proof of Theorem~\ref{two}]If $s < 0$, then $P:L^2(M) \to H^{-s}(M) \to L^2(M)$, where the last inclusion is compact since $-s > 0$, and so $P$ is compact. $P$ is also self-adjoint; $(Pu|v) = (u|Pv)$ for $u,v \in C^\infty(M)$. Since $P$ is bounded $L^2 \to L^2$, we may extend this to $u,v \in L^2(M)$. Since $P$ is then a compact self-adjoint operator on a Hilbert space, the statement about the existence of $L^2$ eigenvalues and eigenvectors follows from general theory. Next we prove that all eigenvectors are of class $C^\infty$. If $Pv = 0$, then $v \in C^\infty(M)$ by elliptic regularity. Otherwise, if $Pv = \lambda v$ for $\lambda \neq 0$, then again by elliptic regularity, $v \in H^s(M)$ since $Pv \in H^s(M)$. Iterating, we have that $v \in H^{sn}(M)$ for all $n \in \Z$, and so $v \in C^\infty(M)$.

if $s > 0$, then we may apply the above results to $\tilde{Q}$. Pick an orthonormal basis for the finite-dimensional space $\ker P$. All such vectors are smooth. We show that the eigenvectors $v_n$ of $\tilde{Q}$ associated to non-zero eigenvalues $\lambda_n$ are eigenvectors for $P$ with eigenvalue $\lambda_n^{-1}$ and form an orthonormal basis of $\im P_{s\to 0}$. This will be sufficient since then we will have an orthonormal basis of 
\[L^2(M) = \ker P\oplus \im P_{s \to 0},\]
which consists entirely of smooth functions.

Since $\tilde{Q}$ is zero on $\ker P$ and the decomposition above is orthogonal, if $\tilde{Q}v_n = \lambda v_n$ for $\lambda \neq 0$, then $v_n \in (\ker P)^\bot = \im P_{0 \to -s}$. Thus, the $v_n$ form an orthonormal basis of $\im P_{s \to 0}$. Furthermore, \[v_n = (I-\Pi_{\ker P})v_n =  P\tilde{Q}v_n = \lambda_nPv_n.\] whence we also have the the $v_n$ are eigenvectors of $P$.

Lastly, if $s=0$, then $P$ is bounded $L^2(M) \to L^2(M)$, and so formal self-adjointness implies self-adjointness.\end{proof}

\begin{lem}\label{sdsd}Let $V$ be a Banach space, and let $U,W$ be closed subspaces of $V$ so that $U\n W = 0$ and $U\oplus W = V$. Then there exists some constant $C$ so that $C\norm{x+y} \geq \norm{x}+\norm{y}$ whenever $x \in U$, $y \in V$. Furthermore, if $X$ is another Banach space, then $T:V \to X$ is bounded iff the restrictions $T|_U$ anad $T|_W$ are.\end{lem}
\begin{proof}$U \oplus W$ is an abstract Banach space under the norm $\norm{x+y} = \norm{x}+\norm{y}$. Indeed, this turns $U \oplus W$ into a normed space. It is complete since if $x_n+y_n$ is Cauchy, then $x_n$ and $y_n$ are individually Cauchy, and so converge because $U,W$ are closed. The map $v=x+y \mapsto x+y$ from $V$ to $U \oplus W$ is certainly well-defined, bijective, and bounded because of the triangle inequality. Thus by the open mapping theorem its inverse is bounded. This statement is equivalent to the first conclusion.

For the second conclusion, one direction is clear. For the other, notice that by boundedness, whenever $x \in U$ and $y \in V$,
\[\norm{T(x+y)} \leq \norm{Tx} + \norm{Ty} \leq \max(\norm{T|_U},\norm{T|_V})(\norm{x}+\norm{y}) \leq C\max(\norm{T|_U},\norm{T|_V})\norm{x+y}.\]\end{proof}

\end{document}
